\documentclass[conference, onecolumn]{IEEEtran}
\IEEEoverridecommandlockouts
\usepackage{cite}
\usepackage{amsmath,amssymb,amsfonts}
\usepackage{graphicx}
\usepackage{textcomp}
\usepackage{xcolor}
\usepackage{algorithm,algpseudocode}
\algrenewcommand\algorithmicindent{0.9em}%
\usepackage{soul}
\usepackage{xspace}
\usepackage{subfigure}


%\newcommand{\todo}[1]{\color{red}\textbf{\hl{#1}}\color{black}\xspace}
\newcommand{\todo}[1]{}

\def\BibTeX{{\rm B\kern-.05em{\sc i\kern-.025em b}\kern-.08em
    T\kern-.1667em\lower.7ex\hbox{E}\kern-.125emX}}
\begin{document}
\title{Graph Partitioning Research Review}


\author{Md Maruf Hossain}
%\address{Univerty of North Carolina at Charlotte}
%\address{University City Blvd, Charlotte}
%\address{North Carolina, 28223}

\maketitle

\begin{abstract}
There are lots of graph analysis perform to extract valuable information from the graph data and to handle the large 
size of graphs, lots of auxiliary utility methods are required. Graph partitioning is a method that can perform 
independently itself and mining the knowledge from the large graph or it can help a lot of other graph analysis to 
ensure a faster output or better memory usage. So, in this article, I will review some great works on the graph partitioning 
that can help us in our future research work.    
\end{abstract}

\section{Hyper Graph Partitioning}
\subsection{Paper Name: Hypergraph-Partitioning-Based Decomposition for Parallel Sparse-Matrix Vector Multiplication} 

\subsubsection{Overview}
The paper is all about the decomposition or partitioning of sparse matrix to reduce the communication volume requirement for 
matrix-vector multiplication. In this paper, they showed two major deficiency of the graph model for decomposing the sparse matrices 
for parallel SpMxV. The first one is that it only used symmetric matrices and the second one is that it does not reflect the 
actual communication requirement. To mitigate these issues, they proposed two hyper-graph partitioning models \textit{column-net} and \textit{row-net}, 
and a multilevel partitioning tool PaToH(\textbf{Pa}rtioning \textbf{To}ols for \textbf{H}ypergraphs) to give a better hypergraph 
partitioning to reduce the communication during matrix-vector multiplication. In this paper, author first discussed how the decomposition 
can be helped the SpMV and then showed the deficiency of the existing model. And then they mentioned their proposed models and tool to solve 
the deficiency. They compare the experimental result of their models with the state-of-art model's experimental results. We can 
split the paper into multiple key points,
\begin{itemize}
\item K-\textit{way} row or column wise hypergraph partitioning for SpMV.
\item Standard graph partitioning model.
\item Deficiency of the previous model.
\item \textit{Column-net} and \textit{row-net} \textit{hypergraph} partitioning model.
\item \textit{Pa}rtitioning \textit{To}ols for \textit{H}yperhraphs(PaToH)
\end{itemize}

\subsubsection{Weakness}
Hyper-graph decomposition runs much slower than the graph based decomposition. \textit{hMetis} shows worst result in compare to the \textit{pMetis} a 
graph based decomposition.

\newpage

\subsection{Paper: Multilevel Hypergraph Partitioning: Applications in VLSI Domain}

\subsubsection{Overview}

\subsubsection{Strength}

\subsubsection{Weakness}


\newpage

\subsection{Paper: Fast Recommendation on Bibliographic Networks}

\subsubsection{Overview}

\subsubsection{Strength}

\subsubsection{Weakness}


\newpage

\subsection{Paper: Hypergraph Partitioning for Multiple Communication Cost Metrics: Model and Methods}

\subsubsection{Overview}

\subsubsection{Strength}

\subsubsection{Weakness}



\end{document}
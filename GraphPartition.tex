\documentclass[conference, onecolumn]{IEEEtran}
\IEEEoverridecommandlockouts
\usepackage{cite}
\usepackage{amsmath,amssymb,amsfonts}
\usepackage{graphicx}
\usepackage{textcomp}
\usepackage{xcolor}
\usepackage{algorithm,algpseudocode}
\algrenewcommand\algorithmicindent{0.9em}%
\usepackage{soul}
\usepackage{xspace}
\usepackage{subfigure}


%\newcommand{\todo}[1]{\color{red}\textbf{\hl{#1}}\color{black}\xspace}
\newcommand{\todo}[1]{}

\def\BibTeX{{\rm B\kern-.05em{\sc i\kern-.025em b}\kern-.08em
    T\kern-.1667em\lower.7ex\hbox{E}\kern-.125emX}}
\begin{document}
\title{Graph Partitioning Research Review}


\author{Md Maruf Hossain}
%\address{Univerty of North Carolina at Charlotte}
%\address{University City Blvd, Charlotte}
%\address{North Carolina, 28223}

\maketitle

\begin{abstract}
There are lots of graph analysis perform to extract valuable information format the graph data and to handle the large 
size of graphs, lots of auxiliary utility methods are required. Graph partitioning is a method that can perform 
independently itself and mining the knowledge from the large graph or it can help a lot of other graph analysis to 
ensure a faster output or better memory usage. So, in this article, I will review some great works on the graph partitioning 
that can help us in our future research work.    
\end{abstract}

\section{Hyper Graph Partitioning}
\subsection{Paper Name: } 


\end{document}
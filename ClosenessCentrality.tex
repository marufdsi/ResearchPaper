\documentclass[conference, onecolumn]{IEEEtran}
\IEEEoverridecommandlockouts
\usepackage{cite}
\usepackage{amsmath,amssymb,amsfonts}
\usepackage{graphicx}
\usepackage{textcomp}
\usepackage{xcolor}
\usepackage{algorithm,algpseudocode}
\algrenewcommand\algorithmicindent{0.9em}%
\usepackage{soul}
\usepackage{xspace}
\usepackage{subfigure}


\newcommand{\todo}[1]{\color{red}\textbf{\hl{#1}}\color{black}\xspace}
%\newcommand{\todo}[1]{}

\def\BibTeX{{\rm B\kern-.05em{\sc i\kern-.025em b}\kern-.08em
    T\kern-.1667em\lower.7ex\hbox{E}\kern-.125emX}}
\begin{document}
\title{A Review on: Incremental Closeness Centrality in Distributed System}


\author{Md Maruf Hossain}
%\address{Univerty of North Carolina at Charlotte}
%\address{University City Blvd, Charlotte}
%\address{North Carolina, 28223}

\maketitle

\section{Closeness Centrality}
The closeness centrality of vartex defines how close 
a vertex to the all other vertices. The closeness centrality of a vertex $u$
can be measured using total number of vertices divided by the the sum of the 
shortest path of all other vertices of the graph $G$ from the particular vertex 
$u$. One can use BFS to calculate the shortest path that will cost $O(n+m)$. 
So, for the total n vertices the overall complexity is $O(n(n+m))$ for the 
offline version of the closeness centrality. But in a dynamic graph an edge can 
be inserted or deleted any time from the graph and to measure the closeness 
centrality of this kind of graph is called incremental closeness centrality.

\section{Contributions}
The authors of these papers~\cite{sariyuce2013streamer, sariyuce2013incremental, sariyuce2015incremental} presented a nobel techinique that can calculate 
the incremental closeness centrality only for the affected vertices. 
Their proposed filter can successfully filter out vertices that require to 
update the centrality score.
\begin{itemize}
\item Lavel-based: if an edge need to insert between two vertices $u$ and $v$ then 
the only vertices need to updated their cenjqwwjtrality score can be filtered out by
$|d_G(s,u) - d_G(s,v)|>1,\ s\in V-{u,v}$
\item They maintain identical vertex list that a vartex can represents a group of vertices 
that are similar to the representative vertex.
\item Biconnected component decomposition(BCD): They calculated biconnected component after insertion 
or deletion of an edge in the graph and proved that it is sufficient to perform SSSPs only on one 
of the biconnected component that contains the edge. Other vertices need to updated their centrality 
score if their representative is a articulation vertex of the biconnected component that is holding 
ther new edge. 
\item \textit{SpMM} formulation: In their work, they try to represent the BFS problem as a 
\textit{SpMM} formulation. They also mentioned that, BFS only handle the vertices for a particular 
lavel, so it is not sufficient for a tradition queue based BFS. 
\end{itemize}

\bibliographystyle{unsrt}
\bibliography{reference}

\end{document}